\usemodule[chart]
\setupFLOWcharts[height=2\lineheight, width=3cm]
\setupinteraction[state=start, color=darkblue, contrastcolor=]
\setupindenting[yes, medium]
\setupcaptions[table][location=top, width=.8\textwidth]
\setupcolors[state=start]

\definecolor[mamu][r=.9, g=0, b=.9 ]
\definecolor[mamu1][r=.3, g=.3, b=.3]

\definecolor[darkyellow][y=.625] % a CMYK color
\definecolor[darkred][r=.625] % a RGB color
\definecolor[darkgray] [s=.925] % a gray scale
\setuptyping[color=darkgreen]
\setuptype[color=darkmagenta, style=bold]
\setupexternalfigures
  [location=default]

\starttext


\startstandardmakeup[rigth]

  {\bfb Learn \CONTEXT~for scientific writing:\\ ---The hard way}
\blank[big]
  by

  Lizardo Reyna\\
  farliz@disroot.org

 \vfill

    
    {\externalfigure [by-sa][width=3cm]}\\
    {\em Approved for free cultural works}\\
  
 % \startuseMPgraphic{color demo}

%      path p ; p := fullcircle xscaled 10cm yscaled 2cm ;
%      fill p withcolor \MPcolor{mamu} ;
     
%      for i=0 step 2 until 20 :
%         draw (0,i) withcolor \MPcolor{darkgrey} ;
%         endfor ;

%         numeric done[][], i, j, n ; n := 0 ;
% forever :
% i := round(uniformdeviate(10)) ; j := round(uniformdeviate(2)) ;
% if unknown done[i][j] :
% drawdot (i*cm,j*cm) ; n := n + 1 ; done[i][j] := n ;
% fi ;
% exitif n = 10 ;
% endfor ;
%\stopuseMPgraphic
%\useMPgraphic{color demo}
  
\stopstandardmakeup


% \startMPpage

%   StartPage ;

%   pickup pencircle scaled 2mm ;
 
% %        fill Page enlarged 2mm withcolor \MPcolor{mamu} ;
        
      
% for i=1 upto 3000:
%      v := uniformdeviate(300) + 1 ;
%      h := uniformdeviate(v) / 900 * normaldeviate ;
%      l := uniformdeviate(900) ;
%      drawdot ((-l,0)--(l,0)) rotated uniformdeviate(180) shifted (h,-v)  
%      withcolor darkgray ;
%      endfor
     
%         picture p, q, t;
%         p := textext("Learn \CONTEXT~for scientific writing:")  xsized 16cm ;
%         t := textext("---The Basics")  xsized 7cm ;
%         q := textext("Lizardo Reyna") xsized 4cm ;
%        % r := textext("Lizardo Reyna") xsized 6cm ;
        
%         draw anchored.rt (p, center rightboundary Page shifted (-3cm,   4cm)) ;
%         draw anchored.rt (t, center rightboundary Page shifted (-12cm,  2cm)) ;
%         draw anchored.rt(q, center rightboundary Page shifted ( -2cm, -12cm)) ; % \MPcolor{darkred} ;


%         draw
% unitsquare shifted (3,.8) xscaled 5cm yscaled 2cm
% withpen pencircle scaled 3.0mm withcolor .625yellow ;
     
%     StopPage ;
%     % draw externalfigure "utm_logo.png" scaled 4cm  shifted (1cm,25cm);

% \stopMPpage


\chapter{---The Basics}

\blank[3cm]

\placecontent

\blank[2cm]

\section{What is ConTeXt?}

\CONTEXT~is a typesetting system developed in 1992 by Hans Hagen from PRAGMA-ADE
in Hasselt, The Netherlands. Is an advanced macro package for creating
professional quality documents in pdf, html and, xml formats. The traditional
word processors such as Writer, Abword, or MS word, may be limited with large and
complex documents and when we need to type many text, mathematical and chemical
formulas, insert figures and create complex tables. \CONTEXT~ is a professional
typesetting software with many commands that will help you make your papers look
like a high quality scientific journal. This will increase your motivation and
creativity in the writing process.
\blank[big]
\CONTEXT~official site:
\blank[big]
  \goto{\tfb https://wiki.contextgarden.net}[url(https://wiki.contextgarden.net)]

\section{How \CONTEXT~works?}

Think in \CONTEXT~as a programming language, this means that you need to
write a source code and then compile the source to get a pdf file.
The file containing the source code must to be named with {\tt .tex}
extension. An useful habit is to create an independent directory for
each document; into this directory or folder the {\tt .tex} file have
to be saved. To avoid potential conflicts the name of the file must not
contain spaces or special characters such as \#, \&,
etc. Underscore can be used to separate words in the file names (e.g. {\tt soil_paper.tex}).



\startFLOWchart[c1]

 
  \startFLOWcell
    \name {file}
    \location {1,1}
    \text {soil_paper.tex}
    \connection [rl] {ctx}
  \stopFLOWcell
   \startFLOWcell
    \name {ctx}
    \location {2,1}
    \text {ConTeXt}
    \connection [rl] {pdf}
    \comment[t]{Compile}
  \stopFLOWcell
   \startFLOWcell
    \name {pdf}
    \location {3,1}
    \text {PDF}
    %\connection [rl] {file}
  \stopFLOWcell


\stopFLOWchart

\FLOWchart[c1]

\CONTEXT~works with commands which are words preceded by a backslash
(\backslash) The {\tt soil_paper.tex} is the file where the commands
will be written and you can use any text editor. However the
TeXworks editor is recommended for beginners and can be downloaded from
the official page
\goto{https://www.tug.org/texworks/}[url(https://www.tug.org/texworks/)].

\section{Try online}

Before install \CONTEXT~and TexWorks and use it in you own machine,
you can follow this guide in the online version in a web browser. Copy
the following link and paste it in your web browser,
\goto{https://live.contextgarden.net}[url(https://live.contextgarden.net)]. In
\in{Fig.}[fig:f1] the online version is showed. You can click the
typeset button and the pdf will be created with Hello World! text.



\placefigure[]
            [fig:f1] {\CONTEXT~online version.}
            {\externalfigure [onlinec][width=\textwidth]}



   \startframedtext[width=\textwidth]
     The examples presented in this chapter (green text) can be copied
     and paste into the textbox in the online version of \CONTEXT.
 \stopframedtext

           
\section{The commands}

The commands have {\em arguments} and {\em options}, although some commands can
be use without arguments and options. The first pair of commands you
need to learn is \type{\starttext \stoptext}. Between these commands
we will write all the content of our document. As is showed in
\in{Fig.}[fig:f1] the sentence Hello world! is between the commands.

All text outside the commands will not appear in the document. The
form \type{\start ... \stop} indicate the beginning and the end of
the commands. In our first document we will write these commands with
some blank spaces between them to avoid confusion with the main text
of the document.

\starttyping
  \starttext

  ....
  
  \stoptext

\stoptyping
             
Now we can write a paragraph. The format of the text in the source code does not
matter, however for optimal readability, the optimal length of the lines in the
source file is about 70 characters per line.

\starttyping
  \starttext
  Lorem ipsum dolor sit amet, consectetuer adipiscing elit.  Donec
  hendrerit tempor tellus.  Donec pretium posuere tellus. 
  nisl, tincidunt et, mattis eget, convallis nec, purus.  Cum sociis
  natoque penatibus et magnis dis parturient montes, nascetur
  ridiculus mus.  Nulla posuere.  Donec vitae dolor.  Nullam tris
  diam non turpis.  Cras placerat accumsan nulla.  Nullam rutrum. 
  vestibulum accumsan nisl.
  \stoptext

\stoptyping

Let's try with two paragraphs. \CONTEXT~knows that there are two
paragraphs because of the space between them.


\starttyping
  \starttext
  Lorem ipsum dolor sit amet, consectetuer adipiscing elit.  Donec
  hendrerit tempor tellus.  Donec pretium posuere tellus.  Proin quam
  nisl, tincidunt et, mattis eget, convallis nec, purus.  Cum sociis
  natoque penatibus et magnis dis parturient montes, nascetur
  ridiculus mus.  Nulla posuere.  Donec vitae dolor.  Nullam tristique
  diam non turpis.  Cras placerat accumsan nulla.  Nullam rutrum.  Nam
  vestibulum accumsan nisl.

  Nullam eu ante vel est convallis dignissim.  Fusce suscipit, wisi
  nec facilisis facilisis, est dui fermentum leo, quis tempor ligula
  erat quis odio.  Nunc porta vulputate tellus.  Nunc rutrum turpis
  sed pede.  Sed bibendum.  Aliquam posuere.  Nunc aliquet, augue nec
  adipiscing interdum, lacus tellus malesuada massa, quis varius mi
  purus non odio.  Pellentesque condimentum, magna ut suscipit
  hendrerit, ipsum augue ornare nulla, non luctus diam neque sit amet
  urna.  Curabitur vulputate vestibulum lorem.  Fusce sagittis, libero
  non molestie mollis, magna orci ultrices dolor, at vulputate neque
  nulla lacinia eros.  Sed id ligula quis est convallis tempor.
  Curabitur lacinia pulvinar nibh.  Nam a sapien.
  \stoptext

\stoptyping

\subsection{Sample text}

\CONTEXT~come with text for training, we can use these paragraphs with
the command \type{\input} following to the name of the file (internal
file). We can use the command several times.

The following files names are possible.

\startitemize[packed, columns, three]
         
\item knuth        
\item tufte        
\item reich        
\item materie      
\item douglas      
\item dawkins      
\item ward         
\item zapf         
\item bryson       
\item davis        
\item thuan        
\item hawking      
\item linden       
\item weisman      
\item montgomery   
\item carrol       
\item schwarzenegger
\item carey        
\item waltham      
\item sapolsky     
\item mcnish       
\item klein        
\item greenfield   
\item poe 
\stopitemize

\starttyping
  \starttext

  \input davis
   \blank
  \input knuth
  
  \stoptext

\stoptyping

We now use the command \type{\blank} to separate two paragraphs with a
white space.

\section{Document structure}

The typical structure of an article is conformed by the following sections:

\startitemize[packed]
\item Title
\item Abstract
\item Introduction
\item Methods
\item Results
\item Discussion
\item Conclusions   
\stopitemize

The references section was intentionally omitted, it will be explained latter.

Two cases are possible; numbered sections and no numbered sections. In
\in{Tab.}[tab:t1] the commands for creating sections are listed.



\placetable[]
           [tab:t1]
           {Commands for document structure. Inside the curly brackets the title of the section must be written}
           {
\bTABLE[frame=off]
\bTR\bTD[style=\bf] Numbered \eTD \bTD[style=\bf] No numbered \eTD  \eTR
\bTR\bTD \type{\chapter{}} \eTD \bTD \type{\title{}} \eTD  \eTR
\bTR\bTD \type{\section{}} \eTD \bTD \type{\subject{}} \eTD  \eTR
\bTR\bTD \type{\subsection{}} \eTD \bTD \type{\subsubject{}} \eTD  \eTR
\bTR\bTD \type{\subsubsection{}} \eTD \bTD \type{\subsubsubject{}} \eTD  \eTR
\eTABLE
           }


 Now we can define the structure of the document as follow.

 \starttyping
   \starttext
   
   \chapter{Title of the paper}
   \subject{Abstract}
   \section{Introduction}
   \section{Methods}
   \section{Results}
   \section{Discussion}
   \section{Conclusions}
   
   \stoptext
 \stoptyping

 We can see in the pdf the number is assigned automatically to each
 numbered section. Now we can fill our paper with the sample text.

 
 \starttyping
   \starttext
   
   \chapter{Title of the paper}
       \subject{Abstract}
           \input ward
       \section{Introduction}
           \input davis
       \section{Methods}
           \input knuth
       \section{Results}
           \input klein
       \section{Discussion}
           \input carey
       \section{Conclusions}
           \input lindey
   
   \stoptext
 \stoptyping

 For code readability we can use tabs to indicate that such section is inside
 other section. These tabs are only for orgnizing the source code and have not
 influence on the pdf.

 \section{Font sizes and styles}

 The font size and style commans must to be used inside curly brackets. The
 commands are showed in Tables \in[tab:t2] and \in[tab:t3].


 \placetable[]
            [tab:t2]
 {Commands to modify the font size}
{ \bTABLE[frame=off, width=5cm]

 \bTR\bTD {\bf Command} \eTD  \bTD {\bf effect} \eTD\eTR
 \bTR\bTD \type{{\tfxx texto}} \eTD  \bTD {\tfxx texto} \eTD\eTR
 \bTR\bTD \type{{\tfx texto}} \eTD  \bTD {\tfx texto} \eTD\eTR
 \bTR\bTD \type{{\tfa texto}} \eTD  \bTD {\tfa texto} \eTD\eTR
\bTR\bTD \type{{\tfb texto}} \eTD  \bTD {\tfb texto} \eTD\eTR
\bTR\bTD \type{{\tfc texto}} \eTD  \bTD {\tfc texto} \eTD\eTR
\bTR\bTD \type{{\tfd texto}} \eTD  \bTD {\tfd texto} \eTD\eTR

\eTABLE
}


\placetable[]
           [tab:t3]
           {Commands to modify the font style}
           {
\bTABLE[frame=off, width=5cm]

%\bTR\bTD {\bfc Comando} \eTD \bTD {\bfc Resultado} \eTD \bTD {\bfc Estilo} \eTD  
%\eTR
\bTR\bTD  \eTD  \eTR

\bTR\bTD \type{{\bf texto}} \eTD \bTD {\tfc \bf texto} \eTD \bTD
         {\tfb Bold face} \eTD \eTR
\bTR\bTD \type{{\it texto}} \eTD \bTD {\tfc \it texto} \eTD \bTD
         {\tfb Italic} \eTD \eTR
 \bTR\bTD \type{{\sl texto}} \eTD \bTD {\tfc \sl texto} \eTD \bTD
          {\tfb Slanted} \eTD \eTR
 \bTR\bTD \type{{\sc texto}} \eTD \bTD {\tfc \sc texto} \eTD \bTD
          {\tfb Smallcaps} \eTD \eTR
\bTR\bTD[width=5cm] \type{{\overstrike texto}} \eTD \bTD {\bfc \overstrike texto} \eTD \bTD
                  {\tfb Overstrike} \eTD \eTR
\bTR\bTD \type{{\ss texto}} \eTD \bTD {\tfc \ss texto} \eTD \bTD
         {\tfb Sans serif} \eTD \eTR
\bTR\bTD \type{{\roman texto}} \eTD \bTD {\tfc \roman texto} \eTD \bTD
         {\tfb Roman} \eTD \eTR
 \eTABLE
           }




           \starttyping
   \starttext
   
   This is a {\tfb Big word} and this a {\bf bold face} style, and the
   italic {\it style}
   
   \stoptext
 \stoptyping


 \section{Lists}

 The command for creating a list is:

 \starttyping
   \startitemize
     ...
   \stopitemize
 \stoptyping

Use the command \type{\item} following the text of the list to create a item, every
 \type{\item} represents one element of the list. The elements can be a simple
 word, line, paragraph or image, see the example.

 \starttyping
   \starttext

   \startitemize
       \item The first element of the list
       \item The second element of the list
       \item The third element of the list
       \item \cdots
   \stopitemize
   
   \stoptext
 \stoptyping


 Now let's use the options for this command. The options must be written
 in brackets at the end of \type{\start...}. In the following example, we
 tell to \CONTEXT~to put numbers instead of symbols in beginning of the list. In
 addition, the option {\it packed} reduce the space between lines.

  \starttyping
   \starttext

   \startitemize[n, packed]
       \item The first element of the list
       \item The second element of the list
       \item The third element of the list
       \item \cdots
   \stopitemize
       
   \stoptext
   \stoptyping

   Check the following examples and analyze the options.


   \starttyping
   \starttext

   \startitemize[n, packed]
       \item The first element of the list
       \item The second element of the list
       \item The third element of the list
       \item \cdots
   \stopitemize

   \input knuth

    \startitemize[n, packed, continue]
       \item The fourth element of the list
       \item The fifth element of the list
       \item The sexth element of the list
       \item \cdots
   \stopitemize
     
   \stoptext
   \stoptyping


If there are to many elements to list, we can divide into columns as follow.

  \starttyping
   \starttext

   \startitemize[n, packed, columns, three]
       \item The first element of the list
       \item The second element of the list
       \item The third element of the list
       \item \cdots
       \item The fourth element of the list
       \item The fifth element of the list
       \item The sexth element of the list
       \item \cdots
   \stopitemize
     
   \stoptext
   \stoptyping

 There are several symbols that can be use in a list. You can change
 the \type{[n]} (numeration) for any of the following alternatives:
 \type{a, A, r, R, 1, 2, 3}. Try and see the results.


 \section{Sub/superscript}

 The command \type{\low{}} and \type{\high{}} put the text as sub or
 superscript form. See the examples.

 \starttyping
 \starttext

The H\low{2}O is the water formula. The area is 80 m\high{2}.
 
 \stoptext
 \stoptyping

 \section{Equations}

 Inline math symbols are written between two \$ \$. \CONTEXT~will print the
 math symbols with an adequate font.


  \starttyping
 \starttext

Solve the following equation $y = a + b - c$
 
 \stoptext
 \stoptyping

If we need isolate the equation in the document, we will use the
\type{\startformula...\stopformula} commands. This will print the
equation in an new line and centered.

 \starttyping
 \starttext

 Solve the following equation

   \startformula
     y = a + b - c
   \stopformula
 
 \stoptext
 \stoptyping

 Note when we use the \type{\start...\stop} form, there is not needed
 the \$ symbol.


 \section{Figures}

 To insert a figure we will use the command \type{\externalfigure
   [ ][width=]}.  In the first pair of brackets you will write the name
 of the image without extension. If the image is in the same folder
 there is not needed write the entire path.

 
 ConTeXt supports the image formats listed below. The image format is determined from the file extension (case insensitive).
 
\startitemize[packed]

\item PDF: File extension .pdf
\item MPS (MetaPost output): File extension .mps or .<digits>
\item JPEG: File extension .jpg or .jpeg
\item PNG: File extension .png
\item JPEG 2000: File extension .jp2
\item JBIG or JBIG2: File extension .jbig, .jbig2, or .jb2

\stopitemize

\starttyping
  \starttext

  This is an inline figure \externalfigure [cow][width=2cm], more text here.
 
  \stoptext
\stoptyping

Captions are important to describe the figure, the command \type{\placefigure}
can use captions and labels. See carefully the brackets types.

\starttyping
  \starttext

  This is an figure with captions and label.

  \placefigure[]
              [fig:f1]
              {This is the caption space. A cow}
              {\externalfigure [cow][width=9cm]}

More text can be write here.
              
  \stoptext
\stoptyping

In the previous example, the first pair of square brackets is for location, let
it empty to led \CONTEXT put it automatically. The second square brackets is for
label, this label will be use latter for cross-reference. The first pair of
curly brackets is for captions, and the second one is for the command.

\stoptext

